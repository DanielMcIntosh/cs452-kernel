\documentclass{article}
\usepackage{hyperref}
\begin{document}

\title{CS 452 Assignment 0}
\author{Brian Forbes \\ 20617899}
\date{\today}

\maketitle

{\huge 1. Partner for Further Assignments}

My partner for future assignments will be Daniel McIntosh (20632796)\\\\

{\huge 2. Building, Running, and Usage}

The repository for the code can be found at \url{https://git.uwaterloo.ca/baforbes/trains}. 

The path to the executable is '/u/cs452/tftp/ARM/baforbes/trains.elf'. The program files can also be found at '/u8/baforbes/cs452/trains'.\\\\
To load the program into RedBoot, use the following command: \begin{verbatim} load -b 0x000218000 -h 10.16.167.5 "ARM/baforbes/trains.elf" \end{verbatim}
To run it, use the following command: \begin{verbatim} go \end{verbatim}
To build the executable, run the following command from the src directory: \begin{verbatim} make \end{verbatim}
To build this document, run the following command from the root directory: \begin{verbatim} pdflatex Assignment0.tex \end{verbatim}

Once run, the program will initialize the display, in addition to beginning a process to set all switches to curved. After that, the following commands are available:
\begin{enumerate}
\item \verb|tr <train_number> <train_speed>|\\
    Sets any train in motion at the desired speed, where speed is between 0 and 14. To stop, set the speed to 0. To turn on lights, add 16 to any speed command.
\item \verb|rv <train_number>|\\
    Reverses any train. Note that this command only works for one train at a time.
\item \verb|sw <switch_number> <switch_direction>|\\
    Throws the given switch to straight (S) or curved (C).
\item \verb|all <switch_direction>|\\
    Throws all switches to straight (S) or curved(C).
\item \verb|q|\\
    Quits the program, and returns to RedBoot.
\end{enumerate}

Various failsafe inputs have also been added, in case program execution unexpectedly fails catastrophically (while continuing to parse input).
\begin{enumerate}
\item \verb|Q|(shift+q) at any time should quit the program.
\item \verb|O|(shift+o) at any time should busy-wait put the solenoid off command, and quit the program. This should be used if the solenoid fails to turn off after a switch command.
\end{enumerate}

{\huge 3 Program Description}\\

Note that each referenced file corresponds to a header file of the same name in the \verb|include| folder (for example, \verb|src/clock.c| to \verb|include/clock.h|)

{\large 3.1 Polling Loop}

The polling loop is in the file \verb|src/trains.c|. In each loop, the following checks occurr:
\begin{enumerate}
\item Read the current time from the 32-bit CLOCK3 register. The logic for this operation is in \verb|src/clock.c|. In addition to the initial read value, various other timing variables are set for latency-tracking output. If 100 milliseconds have elapsed, the bytes to update the clock display is added to the terminal output buffer.
\item If the current time has passed the time at which to turn the solenoid off, add the solenoid off bytes to the train controller output buffer.
\item If the current time has passed the time at which to reverse the currently reversing train, add the reverse command to the train controller output buffer.
\item If the current time has passed the time at which to reaccelerate the currently reversing train, add the reaccelerate command to the train controller output buffer.
\item If the current time has passed the time at which to set the next switch for an \verb|all| command, add the switch command to the train controller output buffer, and set the time for the next switch.
\item If there's a character to output to the terminal, and the terminal UART is ready, put that character into the terminal UART.
\item If there's a character to output to the train controller, and the train controller UART is ready, put that character into the train controller UART.
\item If there's a character to read from the terminal, read it. If the character is a newline, parse the previous command. Logic for command parsing is in \verb|src/terminal.c|
\item If the program is not waiting for any sensor input, or the sensor timeout has passed, add the sensor data request to the train controller output buffer.
\item If there is a byte ready to recieve from the train controller, read it. If all 10 bytes of a request have been read, parse them and output the sensor results to the screen.

\end{enumerate}
{\large 3.2 Command Parsing}

Commands are parsed in \verb|src/terminal.c|. The command is decided based on the first character, and then checked to ensure that the rest of the command is correct. The bytes as a result of the command are output to the train controller buffer. The return code is used to determine if the polling loop should exit.

{\large 3.3 Other Files}

\verb|comio.c| contains UART input/output logic. \verb|clock.c| contains timer input/output logic. \verb|circlebuffer.c| contains the circluar buffer logic.

{\large 3.4 Data Structures}

The main data structure used is a circular buffer. This is used for all input and output. It is implemented as a statically-sized array, with a read and write index. The empty flag disambiguates the full and empty cases (both where read == write). Using a circular buffer allowed me to treat all buffers as functionally infinite, so I could freely add and read data.

The size of 400 was chosen because it allowed me to output my initial terminal, but was small enough that it did not cause memory issues.

There is also an array that keeps track of the speed of all trains, an array to keep track of past sensor data, as well as an array to keep track of recieved sensor data before it is parsed.

Lastly, there is a context structure containing all values meant to be global, such that the main global value is a single pointer.

{\large 3.5 Known Bugs}

\begin{enumerate}
\item Only one train can be reversed at one time. There may also be odd behavior if a train is reversed and sent other commands during the reversing process. As well, if the train speed has not been set before a reverse command, it will not properly re-accelerate.
\item Sending multiple switch commands in very fast succession may not work properly. The program does not enforce a delay between switch commands unless they are a part of an \verb|all| command.
\item Sometimes, there are residual bytes from a sensor query on program startup, offsetting all recieved bytes by 1 byte. I'm pretty sure I fixed this, but due to the non-deterministic nature of this bug, it may still exist.
\item The sensor output does not handle the train controller being off very well. Please ensure that the train controller is on when the program is started, and is not turned off during program execution.
\end{enumerate}

{\large 4. Question Responses}
\begin{enumerate}
\item[i]
    As can be seen in the performance output of the loop, the value MAX (the maximum value over the entire program run) is typically 2500 clock cycles (\textasciitilde5 MS). LMAX (the maximum value during the past 100 MS) is typically around 1500 clock cycles (\textasciitilde3 MS). AVG (the average loop time over the entire program run) is typically around 50 (\textasciitilde0.1MS). As such, since the program only needs to update the clock every 100 MS, it will not miss any clock ticks - the worst case poll time is only \textasciitilde5MS, much less than 100 MS. In addition, it will almost never miss input from the Train Controller, because the loop is generally under 4MS.
\item[ii]
    SNSR (the time taken to reply to an entire sensor query) rests around 31000 clock cycles (\textasciitilde61 MS). FSNR (the time taken to recieve the first byte of a response to a sensor query) rests around 9000 clock cycles (\textasciitilde17MS).

\end{enumerate}

{\large 5. Files in Repo at Time of Last Commit \\(commit 854d786402a86bbea9af98becb2cd8304bf3533a) \\ not including this PDF}
\begin{verbatim}
f978b9288a21c971a5ffddd681f8c250  src/circlebuffer.c
ce4a7222769b3e8726f7ef3f785c6b8d  src/clock.c
53c24877b8ef12668fbd3cc152690458  src/comio.c
a54622187cad51390cce1ddd71d1d884  src/Makefile
a20de032ad8842c6681df68d11200267  src/orex.ld
0522068964eee6a1175ecbf6662b24be  src/terminal.c
beeff5f6addf3f45eae797fc905d4816  src/track_data.c
71eadc380f8d734b8853316af2847ae3  src/trains.c
ba868ea1845b6aa4af4cb1feee528228  lib/libbwio.a
d32dda3f6cd59b210c03d1ed8332c581  include/bwio.h
86d509fa4287eca6fe0498e9370e41d4  include/circlebuffer.h
97d3a62ce76ddfbb35d341db9505e73d  include/clock.h
f763dd3297633a574081637ffa6faa3e  include/comio.h
182adaa2df09ae00d9d354f096ed0dd9  include/terminal.h
1352f3743944badbb8c2399e6fb2ccd4  include/track_data.h
4d93347e509402997aa070c2eb0a4884  include/track_node.h
e979d04e174bfa60630c53718a5dc5cc  include/trains.h
9af226f127c1fd759530cd45236c37b8  include/ts7200.h
\end{verbatim}

\end{document}
