\documentclass{article}
\usepackage{hyperref}
\begin{document}

\title{CS 452 Kernel 2}
\author{Brian Forbes \& Daniel McIntosh \\ 20617899 \& 20632796}
\date{\today}

\maketitle

\section{Building, Running, and Usage}

The repository for the code can be found at \url{https://git.uwaterloo.ca/baforbes/cs452-kernel/tree/k2}.
To build the executable, run the following command from the root directory: \begin{verbatim} make \end{verbatim}
To build this document, run the following command from the root directory: \begin{verbatim} make docs \end{verbatim}

Once run, the program will initialize the kernel, and then create the first user task. This task will then create 3 other tasks: the Rock Paper Scissors (RPS) server, and 2 RPS clients. The clients will play $10$ RPS games against each other, awaiting user input when they recieve replies from the RPS server so the game results are displayed.

\section{Program Description}
Note that each referenced file corresponds to a header file of the same name in the \verb|include| folder (for example, \verb|src/kernel.c| to \verb|include/kernel.h|)

\section{Program Structure}
\subsection{Message Passing}
Each task struct now has a recieve queue (implemented via head and tail pointers). The handling for message passing syscalls is in \verb|sys_handler.c|.
\subsubsection{Send}
    When a task sends a message, it checks if the reciever is already send blocked. If so, it checks the length of the message, and compares it to the length the reciever expects. If these do not match, the return value will be set to an error. It then copies the message into the reciever struct, up to the minimum of the two lengths.
    If the reciever is not send blocked, the sender is added to the send queue of the receiver.
\subsubsection{Receive}
    If the send queue of the task is not empty, the message is immediately copied into the reciever (including the length check as above), and the sender is moved to the reply blocked state.
    If the send queue is empty, the reciever moves into the send blocked state.
\subsubsection{Reply}
    After checks to ensure that the task exists and is reply blocked, as well as the message length check, the message is copied into the reply message pointer, and the reply blocked task is moved into the ready state.
\subsection{Nameserver}
    The nameserver implementation is in \verb|name.c|.
    On startup, the nameserver stores its TID in a global int to ensure that all tasks know it.
    The nameserver stores names in a simple hashtable with linear probing (discussed in the Data Structures section).
    After initializing its storage struct, the nameserver enters a forever loop.
    On each iteration of the loop, it recieves a message, and executes the desired request.
    For WhoIs requests, the name is looked up in the hashtable.
    For RegisterAs, the name is added to the hash table.
\subsection{RPS}
   The RPS implementation is in \verb|rps.c|.
\subsubsection{Server}
   The RPS server stores data in two structs (games, plays) and one int (unpaired).
   We chose to store only one unpaired task at a time, because there can never be two unpaired tasks at once (if there were, they ought to have been paired together already).
   On initialization, the data structures are initialized, and the server registers to the name server.
   It then enters a forever loop.
   On each iteration of the loop, first recieves a message. Thenm it edecutes the desired request.
   For signup requests: If there is no waiting task, it sets this task to the unpaired task.
   If there is a waiting task, it pairs the two tasks (by setting their values in the games array to each other) and replies to each of them.

   For play requests: The opponent is determined via the value in the games array.
   If the opponent has already played their move (determined by the value in the plays array), it compares the two moves, and returns the victory (or quit) status to each task. 
   If the opponent has not yet played, it sets the value in the plays array to the desired move.

   Quit requests are implemented as play requests with a specific move value.
\subsubsection{Client}
    After finding the RPS server TID and signing up for a game, clients play 10 moves. Each move is chosen via taking the low bits of the 40-bit debug clock modulo 3 - effectively randomly. After 10 games, they quit, and then exit.

\subsection{Changes from Kernel 1}
    The architecture from Kernel 1 has not been significantly changed, but various optimizations were added. For example, unneccesary instructions were removed from \verb|asm/activate.s|, the loop in \verb|memcpy| was unrolled, more information was added to aid GCC's optimization, and sequential stack pushes/pops of individual registers were merged into \verb|stmdb| and \verb|ldmia| instructions.
\subsection{New Data Structures}
    The only major new data structure used is the hash table in the name server. We chose a hash table with linear probing because it was relatively small (only a factor $1.3$ time larger than an array of the size of the maximum number of names we expected to store), easy to implement, and was a decent performance increase in the average case. We recognized that most stores and lookups only happened on kernel startup, but still saw benefit to these performance increases.

    The hashtable was implemented as an array, with linear probing. To insert, the item is either put into the slot of the hash of the key, or the next empty slot below that (cyclically).
    Similarly, to look up a key, the slot for the hash of that key is queried.
    If it is not the correct item, the hashtable is probed linearly downwards (cyclically) until the key is found.
    In both operations, the operation fails if the hashtable is full.

    As a hash function, we chose the djb2 function (from http://www.cse.yorku.ca/~oz/hash.html). This is because it is extremely simple, fast, and has good distribution of hashes.
\section{Explanation of Program Output}
The output of the program should be:
\begin{verbatim}
Created RPS Server: 2
Created RPS Client 1: 3
Created RPS Client 2: 4
\end{verbatim}
After this, the output will be 20 lines of the form: \\\"\$(id): I played \$(play), Result: \$(result)\", detailing the results of 10 games.
Each line represents the output from one of the RPS tasks. It contains the ID of the task, the move that task chose, and the result of that task.

Instead of having the RPS server pause after every game, we chose to have the RPS clients pause.
This is to ensure that not only were moves being sent correctly and results were being computed correctly, but the clients were also recieving the correct reply message.
\section{Timing Data}
This data has also been emailed in the requested format, but it is added here for posterity.\\
\begin{tabular}{|c|c|c|c|c|}
\hline
Message length & Caches & Send before Reply & Optimization & Time\\
\hline
4   &    off  &   yes   &      off       &      308\\
64  &    off  &   yes   &      off       &      641\\
4   &    on   &   yes   &      off       &      22\\
64  &    on   &   yes   &      off       &      44\\
4   &    off  &   no    &      off       &      298\\
64  &    off  &   no    &      off       &      635\\
4   &    on   &   no    &      off       &      22\\
64  &    on   &   no    &      off       &      44\\
4   &    off  &   yes   &      on        &      138\\
64  &    off  &   yes   &      on        &      242\\
4   &    on   &   yes   &      on        &      10\\
64  &    on   &   yes   &      on        &      16\\
4   &    off  &   no    &      on        &      137\\
64  &    off  &   no    &      on        &      241\\
4   &    on   &   no    &      on        &      10\\
64  &    on   &   no    &      on        &      16\\
\hline
\end{tabular}

\end{document}
